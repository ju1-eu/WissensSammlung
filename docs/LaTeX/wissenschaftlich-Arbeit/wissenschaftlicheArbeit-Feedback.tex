% Dokumentenklasse
\documentclass[12pt,a4paper,oneside]{article}

% Pakete
% Schriftarten und Typografie
\usepackage{fontspec}  % Ermöglicht die Verwendung von TrueType/OpenType-Schriftarten mit XeLaTeX
\usepackage[T1]{fontenc} % Korrekte Darstellung von Umlauten und Sonderzeichen
\usepackage{textcomp}   % Zusätzliche Schriftzeichen wie Euro-Symbol
\usepackage{microtype}  % Optimiert die Mikrotypografie für bessere Lesbarkeit

% Hauptschriftart einstellen (Tex Gyre Termes, aber anpassbar)
\setmainfont[
    Path = /Users/jan/Library/Fonts/,  % Pfad zu den Schriftarten
    BoldFont = texgyretermes-bold.otf,
    ItalicFont = texgyretermes-italic.otf,
    BoldItalicFont = texgyretermes-bolditalic.otf
]{texgyretermes-regular.otf}

% Monospaced-Schriftart für Quellcode oder ähnliches
\setmonofont{Source Code Pro}


\usepackage[ngerman]{babel}   % Deutsche Sprachunterstützung
\usepackage[german=quotes]{csquotes}

\usepackage{amsmath} % Erweiterte mathematische Funktionalität
\usepackage{amssymb} % Zusätzliche mathematische Symbole
\usepackage{microtype} % Verbesserte Typografie
\usepackage{graphicx} % Für Grafiken
\usepackage{tikz}
\usetikzlibrary{mindmap, shapes, backgrounds}
\usepackage{setspace}
\usepackage[font=small,labelfont=bf]{caption}
\usepackage{subcaption} % Mehrere Unterbilder
\usepackage[left=2.5cm,right=2.5cm,top=2.5cm,bottom=2.5cm]{geometry}
\usepackage{url}

% Tabellen und Listen
\usepackage{booktabs}   % Hochwertige Tabellen
\usepackage{multirow}
\usepackage{makecell}
\usepackage{tabularx}   % Flexible Tabellen
\usepackage{enumitem}   % Anpassen von Aufzählungen und Listen
\setlist[enumerate,1]{label=\alph*)}
\setlist{itemsep=0pt}   % Kein Abstand zwischen Listenelementen

% Literaturverzeichnis
\usepackage[backend=biber, style=ieee, defernumbers=true]{biblatex}  % Verwendung von Biber als Backend und IEEE-Stil (kompatibel mit XeLaTeX)
\addbibresource{references.bib}  % Einbinden der Bibliographie-Datei
\ExecuteBibliographyOptions{backref=true,backrefstyle=three+,url=true,urldate=comp,abbreviate=false,maxbibnames=20}
\DeclareBibliographyCategory{cited}
\let\defaultcite\cite
\renewcommand*\cite[2][]{\addtocategory{cited}{#2}\defaultcite[#1]{#2}}

% Farben und Listings für Quellcode
\usepackage[dvipsnames,svgnames,x11names]{xcolor} % Erweiterte Farbunterstützung
% colors_settings.tex
\providecommand{\definecolor}{}
\definecolor{DodgerBlue4}{rgb}{0.06, 0.31, 0.55}
\definecolor{DarkOrange}{rgb}{1.0, 0.55, 0}
\definecolor{DarkSlateGray}{rgb}{0.18,0.31,0.31}
\definecolor{gray33}{rgb}{0.33,0.33,0.33}
\usepackage{listings}
% listing_settings.tex
\lstset{
    basicstyle=\ttfamily\small,  % Monospaced-Schriftart für Quellcode
    language=C++,  % Programmiersprache
    breaklines=true,  % Zeilenumbruch
    showspaces=false,  % Keine Leerzeichen anzeigen
    showstringspaces=false,  % Keine Leerzeichen in Strings anzeigen
    showtabs=false,  % Keine Tabs anzeigen
    tabsize=4,  % Tabulatorgröße
    captionpos=t,  % Position der Bildunterschrift
    breakatwhitespace=false,  % Zeilenumbruch bei Leerzeichen
    title=\lstname,  % Titel des Listings
    keywordstyle=\color{DodgerBlue4},  % Stil für Schlüsselwörter
    commentstyle=\color{gray33},  % Stil für Kommentare
    stringstyle=\color{DarkOrange},  % Stil für Strings
    %morekeywords={std, cout, endl},  % Zusätzliche Schlüsselwörter
    identifierstyle=\bfseries\color{black},  % Stil für Identifikatoren
    floatplacement=htbp,
    abovecaptionskip=.5\baselineskip,
    belowcaptionskip=.5\baselineskip,
    upquote=true,
    literate={á}{{\'a}}1 {é}{{\'e}}1 {í}{{\'i}}1 {ó}{{\'o}}1 {ú}{{\'u}}1
                {Á}{{\'A}}1 {É}{{\'E}}1 {Í}{{\'I}}1 {Ó}{{\'O}}1 {Ú}{{\'U}}1
                {à}{{\`a}}1 {è}{{\`e}}1 {ì}{{\`i}}1 {ò}{{\`o}}1 {ù}{{\`u}}1
                {À}{{\`A}}1 {È}{{\'E}}1 {Ì}{{\`I}}1 {Ò}{{\`O}}1 {Ù}{{\`U}}1
                {ä}{{\"a}}1 {ë}{{\"e}}1 {ï}{{\"i}}1 {ö}{{\"o}}1 {ü}{{\"u}}1
                {Ä}{{\"A}}1 {Ë}{{\"E}}1 {Ï}{{\"I}}1 {Ö}{{\"O}}1 {Ü}{{\"U}}1
                {â}{{\^a}}1 {ê}{{\^e}}1 {î}{{\^i}}1 {ô}{{\^o}}1 {û}{{\^u}}1
                {Â}{{\^A}}1 {Ê}{{\^E}}1 {Î}{{\^I}}1 {Ô}{{\^O}}1 {Û}{{\^U}}1
                {œ}{{\oe}}1 {Œ}{{\OE}}1 {æ}{{\ae}}1 {Æ}{{\AE}}1 {ß}{{\ss}}1
                {ű}{{\H{u}}}1 {Ű}{{\H{U}}}1 {ő}{{\H{o}}}1 {Ő}{{\H{O}}}1
                {ç}{{\c c}}1 {Ç}{{\c C}}1 {ø}{{\o}}1 {å}{{\r a}}1 {Å}{{\r A}}1
                {€}{{\EUR}}1 {£}{{\pounds}}1 {~}{{\textasciitilde}}1 {-}{{-}}1
}
    

% Fußnoten
\usepackage{footnote}   % Verbesserte Fußnotenverwaltung
\usepackage{fnpct}      % Verwaltung der Fußnoten und Interaktion mit Satzzeichen
\setfnpct{after-punct-space={-.2em}}

% Globale Einstellung für alle Listen
\setlist{noitemsep, topsep=0pt}

% Formatierung
\onehalfspacing
\renewcommand{\normalsize}{\fontsize{12pt}{14pt}\selectfont}
\renewcommand{\large}{\fontsize{14pt}{17pt}\selectfont}

% Umbenennen der Verzeichnisse
\renewcommand{\listfigurename}{Abbildungen}
\renewcommand{\listtablename}{Tabellen}
\renewcommand{\refname}{Literatur}
\renewcommand{\figurename}{Abb.}
\renewcommand{\tablename}{Tab.}

% Hyperlinks und Querverweise
\usepackage[hidelinks]{hyperref}  % Versteckte Links (ohne Umrandung)
\usepackage[ngerman]{cleveref}    % Intelligente Querverweise
\hypersetup{
    colorlinks=true,
    linkcolor=meinblue,
    filecolor=meinblue,      
    urlcolor=meinblue,
    citecolor=meinblue, % Farbe für Literaturverweise
    pdftitle={Redaktionelles Feedback: LaTeX-Vorlage},
    pdfpagemode=FullScreen,
}


\title{Umfassendes Redaktionelles Feedback: LaTeX-Vorlage für wissenschaftliche Ausarbeitung}
\author{}
\date{}

\begin{document}

\maketitle

\tableofcontents

\section{Struktureller Aufbau einer schriftlichen Ausarbeitung}

% Beispiel für eine Fußnote mit URL
Google\footnote{\url{https://www.google.com}}.

Tipps für schriftliche Ausarbeitungen\footnote{\url{https://osm.hpi.de/theses/tipps}}.

% Beispiel für eine Zitation
OpenAI \cite{statista2024openai} ist ein führendes KI-Unternehmen, das für Anwendungen wie ChatGPT\footnote{\url{https://chatgpt.com/?model=gpt-4}} und Dall-E bekannt ist.


\subsection{Allgemeine Struktur}
\begin{enumerate}
    \item Titelei
    \item Textteil
    \item Anhang
\end{enumerate}

\subsection{Kurzfassung (Abstract)}
\textbf{Zweck}: Prägnante Darstellung der wichtigsten Inhalte
\begin{itemize}
    \item \textbf{Eigenschaften}:
    \begin{itemize}
        \item Objektiv und kurz
        \item Klare Sprache und Struktur
        \item Enthält alle wesentlichen Fragestellungen, Ansätze und Erkenntnisse
    \end{itemize}
    \item \textbf{Wichtige Aspekte}:
    \begin{itemize}
        \item Unabhängig vom Rest der Arbeit
        \item Enthält alle wichtigen Inhalte (Spoiler, kein Teaser)
        \item Keine Meta-Formulierungen oder Quellenangaben
    \end{itemize}
\end{itemize}

\subsection{Einleitung}
\textbf{Inhalt}:
\begin{itemize}
    \item Vorstellung und Motivation des Themas
    \item Auflistung der Fragestellungen
    \item Überblick über die Problembehandlung
\end{itemize}
\textbf{Wichtige Aspekte}:
\begin{itemize}
    \item Relevanz und Aktualität des Themas darstellen
    \item Zentrale Fragen auflisten
    \item Kontext und Hintergrundwissen vermitteln
\end{itemize}

\subsection{Grundlagen}
\textbf{Zweck}: Definition zentraler Begriffe und Einführung wichtiger Konzepte\\
\textbf{Fokus}: Problembezogene Aspekte hervorheben\\
\textbf{Wichtige Aspekte}:
\begin{itemize}
    \item Problemorientierte Definitionen statt Lexikon-Paraphrasierungen
    \item Diskussion von Möglichkeiten, Herausforderungen und Limitierungen
    \item Abgrenzung des Themas
\end{itemize}

\subsection{Hauptteil}
\textbf{Inhalt}: Abhängig von Thema und Vorgehensweise\\
\textbf{Mögliche Bestandteile}:
\begin{itemize}
    \item Beschreibung eines Entwurfs
    \item Umsetzung und Umsetzungsalternativen
    \item Evaluierung und Bewertung
\end{itemize}

\subsection{Schlussteil}
\textbf{Inhalt}:
\begin{itemize}
    \item Zusammenfassung der Kernaussagen
    \item Beantwortung der Forschungsfragen
    \item Kritische Betrachtung der eigenen Arbeit
    \item Ausblick auf zukünftige Arbeiten
\end{itemize}
\textbf{Wichtige Aspekte}:
\begin{itemize}
    \item Keine Quellenangaben
    \item Raum für eigene Sichtweisen und Meinungen
\end{itemize}

\subsection{Literaturverzeichnis und Quellenangaben}
\textbf{Grundsätze}:
\begin{itemize}
    \item Nur zitierte Quellen aufführen
    \item Einheitliche Formatierung der Verweise
\end{itemize}
\textbf{Wichtige Aspekte}:
\begin{itemize}
    \item Verwendung von Literaturverwaltungsprogrammen
    \item Vollständige Angaben zu jeder Quelle
    \item Kritische Bewertung von Internetquellen
    \item Klare Abgrenzung zwischen Literaturaussagen und eigener Auslegung
\end{itemize}

\begin{quote}
\glqq Die Kurzfassung ist keine Einleitung. Sie muss unabhängig von der Arbeit betrachtet werden.\grqq{}
\end{quote}

\section{Inhalt \& roter Faden}

\subsection{Allgemeine Prinzipien}
\begin{itemize}
    \item Nachvollziehbare Argumentation für Lesende
    \item Relevanz jedes Inhalts für das Ergebnis muss klar sein
    \item Betrachtung von Argumenten aus verschiedenen Perspektiven
    \item Inhalte müssen relevant, richtig, sachlich und nachvollziehbar sein
    \item Fähigkeit zur Unterscheidung zwischen Wichtigem und Unwichtigem demonstrieren
\end{itemize}

\subsection{Roter Faden}
Sichtbar auf mehreren Ebenen:
\begin{enumerate}
    \item Gliederungsebene
    \item Gedankliche Ebene
    \item Sprachliche Ebene
\end{enumerate}

\section{Gliederung}

\subsection{Wichtige Aspekte}
\begin{itemize}
    \item \textbf{Logischer Aufbau} der Gliederung
    \item Inhaltliche Schwerpunkte in der Struktur erkennbar
    \item Angemessene Gliederungstiefe (nicht zu zersplittert)
    \item Gleicher Abstraktionslevel bei gleicher Gliederungstiefe
    \item Bezug zur Fragestellung stets erkennbar
\end{itemize}

\subsection{Strukturierung}
\begin{itemize}
    \item Verwendung von \verb|\section|, \verb|\subsection|, \verb|\paragraph|
    \item Problem vor der Lösung darstellen
    \item Einleitung, Hauptteil, Schluss auch in einzelnen Abschnitten
    \item Aufeinander aufbauende Absätze
    \item Vermeidung von \glqq Micro-Absätzen\grqq{}
\end{itemize}

\section{Argumentation}

\subsection{Grundsätze}
\begin{itemize}
    \item Objektive Nachprüfbarkeit von Aussagen
    \item Sachliche Begründung jeder Entscheidung
    \item Korrekte, genaue und vollständige Darstellung von Thesen, Argumenten und Beispielen
    \item Kritische Diskussion des Problems und möglicher Nachteile
\end{itemize}

\subsection{Wichtige Punkte}
\begin{itemize}
    \item Ausreichende Menge relevanter, korrekter und aktueller Argumente
    \item Verknüpfung von Informationen
    \item Vermeidung von Allaussagen und Pauschalisierungen
    \item Nachvollziehbare Begründungen statt bloßer Behauptungen
    \item Vorsicht bei der Verwendung von Werbebegriffen
\end{itemize}

\section{Gewichtung}

\subsection{Prinzipien}
\begin{itemize}
    \item Wichtige Aussagen erhalten mehr Platz
    \item Minimierung von Wiederholungen und Trivialitäten
    \item Verweis auf externe Quellen für allgemein bekannte Informationen
    \item Angemessene Würdigung zeitintensiver Forschungstätigkeiten
\end{itemize}

\begin{quote}
\glqq Lesende mögen keine anderthalb Seiten, die keinen Beitrag liefern\grqq{}
\end{quote}


\section{Weitere wichtige Formalien}

\subsection{Grundprinzip}
\begin{itemize}
    \item Einheitlichkeit, Übersichtlichkeit und Systematik
\end{itemize}

\subsection{Umgang mit Fachbegriffen und Fremdwörtern}
\begin{itemize}
    \item \textbf{Erste Verwendung}: Kenntlich machen (z.B. mit \verb|\emph|) und kurz erläutern
    \item Bevorzugung deutscher Begriffe, wenn möglich
    \item Erklärung durch Übersetzung oder Nebensatz/Fußnote
\end{itemize}

\subsection{Abkürzungen}
\begin{itemize}
    \item Sparsame und übliche Verwendung
    \item Aufschlüsselung bei erster Verwendung
    \item Keine Abkürzungen am Satzanfang
    \item Korrekte Formatierung mit Leerzeichen (z.B.~u.\,a.)
\end{itemize}

\subsection{Struktur und Gliederung}
\begin{itemize}
    \item Mindestens zwei Unterpunkte bei Verwendung von Unterpunkten
    \item Vermeidung von direkter Subsection nach Section ohne Text
    \item Verwendung von \glqq Topic Sentences\grqq{} am Anfang von Sections
\end{itemize}

\subsection{Quellenangaben und URLs}
\begin{itemize}
    \item URLs als Fußnote oder Referenz, nicht im Fließtext
    \item Bei Produktnennungen: URL als Fußnote
    \item Verwendung von \verb|\url|-Umgebung in LaTeX
\end{itemize}

\subsection{Schreibstil}
\begin{itemize}
    \item Vermeidung der ersten Person Singular (außer bei eigenen Leistungen)
    \item Aufzählungen nur bei genaueren Erklärungen, sonst Fließtext
    \item Keine zusätzlichen Formatierungen in Überschriften
\end{itemize}

\subsection{LaTeX-spezifische Hinweise}
\begin{itemize}
    \item Absatztrennung durch Leerzeile, nicht \verb|\\|
    \item Verwendung von \verb|--| für Gedankenstriche
    \item Nutzung von \verb|~| für geschützte Leerzeichen
    \item Referenzen für alle Quellen im Quellenverzeichnis
    \item Korrekte Formatierung von Firmen-/Organisationsnamen in .bib-Dateien
\end{itemize}

\subsection{Korrektur und Überarbeitung}
\begin{itemize}
    \item \textbf{Mehrfache Überarbeitung vor Abgabe}
    \item Beseitigung von Wiederholungen, Umstellung von Abschnitten
    \item Überprüfung des roten Fadens und der Argumentation
    \item Externe Rechtschreibkontrolle
    \item Überprüfung von Layout, Referenzen und Literaturverzeichnis
\end{itemize}

\begin{quote}
\glqq Eine gute schriftliche Ausarbeitung braucht eine gute Argumentation und eine gute Schlussüberarbeitung.\grqq{}
\end{quote}

\subsection{Literaturverzeichnis}
\begin{itemize}
    \item Erstellung als .bib-Datei
    \item Korrekte Formatierung von Autorennamen und Organisationen
    \item Trennung mehrerer Autoren mit \glqq and\grqq{}, nicht Komma
\end{itemize}

% Literaturverzeichnis
\newpage
\phantomsection % Ermöglicht es, dass das Inhaltsverzeichnis den richtigen Link enthält
\addcontentsline{toc}{section}{Literatur} % Fügt "Literatur" dem Inhaltsverzeichnis als Abschnitt hinzu
\printbibliography % Zeigt alle zitierten Einträge an

\end{document}